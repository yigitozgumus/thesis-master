%!TEX root = ../2019_7_Ozgumus_Semsi_Yigit.tex

\begingroup

In this work, we tackled a problem of building an anomaly detection framework that utilizes a 
generative adversarial network and encoder networks. In particular, we investigated different 
GAN based anomaly detection and feature learning approaches and analyzed their corresponding 
performance with our SEM image dataset. Based on the analysis regarding the stabilization and 
convergence problems experienced by the networks, a modified architecture is proposed. In order to support 
our analyses, an ablation study is performed to mitigate the generator instabilities and the 
effect of training networks in an adversarial setting concurrently or sequentially is tested through 
an incremental model proposition.

Experiments presented in chapter \ref{chap:expres} show that GAN based anomaly detection methods that 
are tested on the SEM Image dataset gives a contradictory performance scores with respect to the visual 
quality of the reconstructions. Even in model configurations which no convergence issue or mode collapse problem 
occurs, visual reconstruction imperfections cause model's performance to not pass a $0.6$ threshold AUROC score.
Models that inherit the adversarial training aspect of GANs but uses an autoencoder architecture in 
their generator networks produced better reconstructions and achieved a higher performance in terms 
of both AUROC score and the overall precision / recall capacity. These observations point out the importance 
of adversarial feature learning process that take effect in these models and how it affects the overall success 
of the anomaly detection quality.

Proposed model which addresses the issues detected in the experiments produced a better performance 
than the other GAN based models. Reconstructions obtained by the model showed that separating the networks 
to stabilize the training procedure improved the visual resemblance while there is still a room to improve 
for the quality of the reconstructions. In particular noisy reconstructions of some of the patches has a high 
chance of being identified as an anomalous sample. Experiments also showed that computing the anomaly 
representation using latent dimensional space instead of the image space can decrease the impact of this problem. 
But the performance improvements are not significant and we speculate that main factor to improve this approach 
is also lies in the quality of the reconstructions.

\endgroup
 
