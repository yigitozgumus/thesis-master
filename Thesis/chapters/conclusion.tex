%!TEX root = ../2019_7_Ozgumus_Semsi_Yigit.tex

\begingroup

	
In this work, we tackled a problem of building an anomaly detection framework that utilizes a 
GAN and encoder networks. In particular, we investigated different 
GAN based anomaly detection and feature learning approaches and analyzed their corresponding 
performance with our SEM image dataset. Based on the analysis regarding the stabilization and 
convergence problems experienced by the networks, modified architecture is proposed. In order to support 
our analyses, an ablation study is performed to mitigate the generator instabilities and the 
effect of training networks in an adversarial setting concurrently or sequentially is tested through 
an incremental model proposition.

Experiments presented in Chapter \ref{chap:expres} show that GAN based anomaly detection methods that 
are tested on the SEM Image dataset gives a contradictory performance scores with respect to the visual 
quality of the reconstructions. Even in model configurations which no convergence issue or mode collapse problem 
occurs, visual reconstruction imperfections cause model's performance to not pass a $0.6$ threshold AUROC score.
Models that inherit the adversarial training aspect of GANs but uses an autoencoders in 
their generator networks produced better reconstructions and achieved a higher performance in terms 
of both AUROC score and the overall precision / recall capacity. These observations point out the importance 
of adversarial feature learning process that take effect in these models and how it affects the overall success 
of the anomaly detection quality.

Proposed model which addresses the issues detected in the experiments produced a better performance 
than the other GAN based anomaly detection models. Reconstructions obtained by the model showed that 
separating the networks to stabilize the training procedure improved the visual resemblance while there 
is still a room to improve for the quality of the reconstructions. In particular noisy reconstructions of 
some of the patches has a high chance of being identified as an anomalous sample. Observation of autoencoder 
variant models such as GANomaly and Skip-GANomaly also showed that computing the anomaly representation using 
latent dimensional space instead of the image space can decrease the impact of the problem of reconstruction quality. 
Model proposed based on this observations produced an AUROC score above $0.7$ in the experiments. We speculate 
that main factor to further improve this approach also depends on the quality of the reconstructions.

As the future work which can be built on top the results of this thesis, it is possible to investigate 
further open issues. Energy based GANs are relatively new topic and the 
quality of the reconstructions has a potential for improvement. Functionality of the autoencoder based 
discriminator network can also be extended to learn the joint distribution of the image and latent 
representation like in BiGAN (see Section \ref{sec:bigan}) to investigate how energy based adversarial 
loss impact on the concurrent learning of bidirectional mapping. 

Moreover, acquiring a relevant data for a problem is still an important issue in all types of machine 
learning task. Apart from its generation capabilities and its contribution to the adversarial feature 
learning process, GANs and its autoencoder variants can be used for the data augmentation task.
Particularly \cite{DBLP:journals/corr/abs-1808-07632} proposes an oversampling method for infrequent 
normal samples which focuses on to improve the false positive rate for unsupervised anomaly detection tasks. 

\endgroup
 
