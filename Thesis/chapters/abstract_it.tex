%!TEX root = ../2019_7_Ozgumus_Semsi_Yigit.tex

\begingroup

L'individuazione di anomalie è un campo sempre più popolare nella visione artificiale ed è una sfidaproblema riguardante un'ampia varietà di domini applicativi. Come i metodi di raccolta e utilizzoi dati crescono, la qualità e le prestazioni dei sistemi che ne traggono beneficiopiù dipendente dai dati che usano. Considerando questi sistemi si basano sulla qualità dei dati,la rilevazione delle anomalie gioca un ruolo cruciale per rilevare anomalie che possono danneggiare il sistema selasciato sconosciuto. Negli scenari in cui i dati anormali sono un evento raro o semplicemente costosoda osservare, gli approcci basati sulla ricostruzione sono usati per modellare i dati normali da impararedifferenziare le anomalie quando incontrate. Le reti del contraddittorio generativo sono un metodo relativamente nuovoche viene utilizzato per generare dati nuovi e inosservati in modo non controllato eil suo adattamento al quadro di rilevamento delle anomalie basato sulla ricostruzione è promettente.

Questa tesi fornisce un modello per prevedere le regioni anomale sulle immagini del microscopio elettronico a scansione (SEM)
di materiali nanofibrati utilizzando un framework basato su reti generative adversarial (GAN) e reti di encoder.
La soluzione proposta apprende la mappatura bidirezionale tra l'immagine e la sua dimensione latente usando il contraddittorio
formazione del generatore della rete generativa avversaria e della rete di encoder.

Dimostriamo che il nostro approccio produce risultati migliori rispetto ad altri framework di rilevamento delle anomalie basati su GAN formati con il nostro
set di dati. Mentre l'aumento delle prestazioni è un piccolo margine, il nostro metodo mostra una migliore ricostruzione visiva dei dati.
\endgroup