%!TEX root = ../2019_7_Ozgumus_Semsi_Yigit.tex

\begingroup

L'individuazione di anomalie è un campo sempre più popolare nella visione artificiale ed è una sfida
problema riguardante vari domini applicativi. Come i metodi di raccolta e utilizzo
i dati crescono, la qualità e le prestazioni dei sistemi di cui beneficiamo nella nostra vita quotidiana diventano
più dipendente dai dati. Considerando questi sistemi si basa sulla qualità dei dati, il rilevamento delle anomalie
gioca un ruolo cruciale per aiutare le loro funzionalità. In scenari in cui i dati anormali sono un evento raro
o semplicemente costosi da osservare, gli approcci basati sulla ricostruzione sono usati per modellare i dati normali da apprendere
per differenziare le anomalie quando incontrate. Le reti del contraddittorio generativo sono un metodo relativamente nuovo
che viene utilizzato per generare dati nuovi e non osservati in modo non supervisionato e la sua adozione alla ricostruzione basata
il quadro di rilevamento delle anomalie è promettente.

Questa tesi fornisce un modello per prevedere le regioni anomale sulle immagini del microscopio elettronico a scansione (SEM)
di materiali nanofibrati utilizzando un framework basato su reti generative adversarial (GAN) e reti di encoder.
La soluzione proposta apprende la mappatura bidirezionale tra l'immagine e il suo spazio di dimensione latente usando il contraddittorio
formazione del generatore della rete generativa avversaria e della rete di encoder.

Dimostriamo che il nostro approccio ottiene risultati migliori rispetto ad altri framework di rilevamento delle anomalie basati su GAN addestrati con i nostri
set di dati. Mentre l'aumento delle prestazioni è un piccolo margine, il nostro metodo mostra una migliore ricostruzione visiva dei dati.

\endgroup