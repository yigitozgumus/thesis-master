%!TEX root = ../2019_7_Ozgumus_Semsi_Yigit.tex

\begingroup

La rivelazione di anomalie \'e un argomento sempre pi\'u popolare nel campo della visione artificiale e 
frattanto costituisce una sfida riguardo a un'ampia variet\'a di dominio dell'appliazione. 
Aree come la sicurezza informatica, la finanza, l'industria e altri hanno adattato i loro metodi di 
soluzione utilizzando i modelli basati sui dati. 
Considerando questi metodi che si basano sulla qualit\'a dei dati, la rilevazione delle anomalie gioca 
un ruolo cruciale per rilevare anomalie nei dati che potrebbero danneggiare il sistema se lasciato indefinito. 
Ad esempio vengono analizzati i dati finanziari relativi alle spese dei clienti e i loro modelli 
di acquisto per svelare dei frodi potenziali \cite{finance_anomaly}. 
I sistemi industriali utilizzano il rilevamento delle anomalie sui dati sensoriali provenienti dall'apparato 
per monitorare la qualit\'a ed esigenze di manutenzione \cite{inproceedings_industry}.

Negli scenari in cui i dati anormali ci mettono un caso raro o semplicemente costosi da osservare, 
gli approcci basati sulla ricostruzione vengono usati per modellare i dati normali per imparare a 
differenziare le anomalie quando incontrate. 
Gli autoencoder sono stati ampiamente utilizzati per la ricostruzione del rilevamento degli anomalie 
in ultimi anni. Generative adversarial network (GAN) \'e un modelli relativamente nuovo che viene utilizzato 
per generare dati nuovi e non osservati in modo non supervisionato e il suo adattamento alle 
impostazioni di rilevamento delle anomalie basate sulla ricostruzione \'e promettente.


Questa tesi propone un modello per prevedere le regioni anormale sulle immagini del microscopio 
elettronico a scansione (SEM) di materiali nanofibrosi che combinano GAN e un encoder. 
La soluzione proposta si istruisce sulla mappatura bidirezionale tra l’immagine e il 
suo spazio di dimensione latente utilizzando l’allenamento contraddittorio della rete 
del generatore e della rete di encoder.

Dimostriamo che il nostro approccio produce risultati migliori rispetto ad altri 
framework di rilevamento delle anomalie basati su GAN formati con il set di dati dell'immagine SEM. 
Mentre l'aumento delle prestazioni \'e un piccolo margine, il nostro metodo ottiene una 
migliore ricostruzione visiva dei dati.
Il nostro lavoro suggerisce che i metodi basati sulla ricostruzione giocano un ruolo fondamentale 
nel rilevamento delle anomalie con i dati delle immagini.


\endgroup