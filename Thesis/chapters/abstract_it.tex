%!TEX root = ../2019_7_Ozgumus_Semsi_Yigit.tex

\begingroup

L'individuazione di anomalie è un campo sempre più popolare nella visione artificiale ed è una sfida
problema riguardante vari domini applicativi. Come i metodi di raccolta e utilizzo
i dati crescono, la qualità e le prestazioni dei sistemi di cui beneficiamo nella nostra vita quotidiana diventano
più dipendente dai dati. Considerando questi sistemi si basa sulla qualità dei dati, il rilevamento delle anomalie
svolge un ruolo cruciale per rilevare anomalie che potrebbero danneggiare il sistema se non vengono scoperte. In scenari dove
dati anormali sono eventi rari o semplicemente costosi da osservare, approcci basati sulla ricostruzione
sono usati per modellare i dati normali per imparare a distinguere le anomalie quando incontrate. Contraddittorio generativo
le reti sono un metodo relativamente nuovo che viene utilizzato per generare dati nuovi e non osservati in modo non controllato e
la sua adozione alla ricostruzione basata sul quadro di rilevamento delle anomalie è promettente.

Questa tesi fornisce un modello per prevedere le regioni anomale sulle immagini del microscopio elettronico a scansione (SEM)
di materiali nanofibrati utilizzando un framework basato su reti generative adversarial (GAN) e reti di encoder.
La soluzione proposta apprende la mappatura bidirezionale tra l'immagine e la sua dimensione latente usando il contraddittorio
formazione del generatore della rete generativa avversaria e della rete di encoder.

Dimostriamo che il nostro approccio produce risultati migliori rispetto ad altri framework di rilevamento delle anomalie basati su GAN formati con il nostro
set di dati. Mentre l'aumento delle prestazioni è un piccolo margine, il nostro metodo mostra una migliore ricostruzione visiva dei dati.
\endgroup