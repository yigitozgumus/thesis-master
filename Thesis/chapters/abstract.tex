%!TEX root = ../2019_7_Ozgumus_Semsi_Yigit.tex

\begingroup

Anomaly detection is an increasingly popular field in computer vision and also is a challenging 
problem concerning a wide variety of application domains.
Areas like cyber security, finance, industry and others have tailored their solution methods to 
utilize data driven models. Considering these methods rely on the quality of data, 
anomaly detection plays a crucial role to detect abnormalities in the data that may harm the system if 
left undiscovered. For example in finance, expenditure history of the customers and their shopping 
patterns are analyzed to detect possible frauds \cite{finance_anomaly}. Industrial systems 
use anomaly detection on the sensory data coming from the equipment to monitor the quality and 
maintenance needs \cite{inproceedings_industry}.

In scenarios where the abnormal data is a rare occurrence or simply costly 
to be observed, reconstruction based approaches are used to model the normal data to learn to 
differentiate anomalies when encountered. Autoencoders have been heavily used for reconstruction based 
anomaly detection in recent years. Generative adversarial network (GAN) is a relatively new model 
that allows to generate new and unobserved data in an unsupervised way and 
its adaptation to reconstruction based anomaly detection setting is promising.

This thesis provides a model to predict the anomalous regions on the Scanning Electron Microscope 
(SEM) images of nanofibrous materials combining a GAN and encoder networks. 
Proposed solution learns the bidirectional mapping between the image 
and its latent dimension space using the adversarial training of the generator network and the 
encoder network. 

We show that our approach produces better results than other GAN based anomaly detection frameworks 
trained with the SEM image dataset. While the performance increase is a small margin, our method shows better 
visual reconstructions of the data. Our work suggests that reconstruction based methods plays a fundamental
role in anomaly detection with image data.
\endgroup