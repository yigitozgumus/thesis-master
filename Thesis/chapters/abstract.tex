%!TEX root = ../2019_7_Ozgumus_Semsi_Yigit.tex

\begingroup

Anomaly detection is an increasingly popular field in computer vision and is a challenging 
problem concerning wide variety of application domains. As the methods of gathering and utilizing 
data grows, quality and the performance of the systems we benefit become
more dependent on the data they use. Considering these systems rely on the quality of data, 
anomaly detection plays a crucial role to detect abnormalities that may harm the system if 
left undiscovered. In scenarios where the abnormal data is a rare occurrence or simply costly 
to be observed, reconstruction based approaches are used to model the normal data to learn to 
differentiate anomalies when encountered. Generative adversarial networks is relatively new method 
that is used to generate new and unobserved data in an unsupervised way and 
its adaptation to reconstruction based anomaly detection framework is promising.

This thesis provides a model to predict the anomalous regions on the Scanning Electron Microscope 
(SEM) images of nanofibrous materials using a framework based on generative adversarial networks 
(GAN) and encoder networks. Proposed solution learns the bidirectional mapping between the image 
and its latent dimension space using the adversarial training of the generator network and the 
encoder network. 

We show that our approach produces better results than other GAN based anomaly detection frameworks 
trained with our dataset. While the performance increase is a small margin, our method shows better 
visual reconstructions of the data.
\endgroup