%!TEX root = ../2019_7_Ozgumus_Semsi_Yigit.tex

\begingroup

%% Anomaly Detectioni anlat abstracten acip
Anomaly detection is an increasingly popular field in computer vision and is a challenging 
problem concerning wide variety of application domains such as cyber security, healthcare, 
finance and Internet of Things (IoT) \cite{Chandola:2009:ADS:1541880.1541882}. As the methods 
of obtaining and utilizing potential use cases for data grows, quality, maintenance and 
the performance of the systems we benefit become more dependent on the provided data. 
Anomaly detection plays a crucial role to preserve the utilization of these systems considering 
these systems rely on the quality and the accuracy of data. Anomalies can be defined as the 
samples that don't follow the characteristic behaviors of the defined notion of normality 
for that data \cite{Chandola07anomalydetection:}. Main importance of anomaly detection comes 
from the fact that abnormalities in the data may often be translated to critical and 
actionable information. 
%% Rec based methodu acikla ve bir iki ornek ver

One of the main problems of solving anomaly detection tasks is to obtain data. 
Depending on the domain, anomaly might be a very rare occurrence or it might be costly for the 
system to replicate the conditions of the anomalous case. Reconstruction based methods for anomaly 
detection focus on computing the representation for normal portion data, and uses this representation 
to differentiate the normal and anomalous samples. This method is used when there is an imbalance 
between normal and anomalous data sample size in the dataset. Main premise is that classifying 
data sample with higher dimensional feature space such as images, as anomalous or normal, 
becomes easier if the underlying data of interest lies in a subspace of lower dimensionality 
than its original form \cite{Beyer:1999:NNM:645503.656271}. 
Hence we can model the data in a lower dimensional complexity and find an appropriate way to reconstruct 
to its original higher-dimensional representation. Obtaining an approximation of the normal data using 
this reconstruction does not give the same type of result when computed with an anomalous sample. 
Autoencoder networks (see Section \ref{sec:ae}) are heavily used for reconstruction based anomaly 
detection tasks in recent years. They consist of an encoder and decoder network. Encoder 
network is trained to learn the latent representation and decoder network is trained to obtain 
the reconstruction to model the normality of data with no anomalies. Over the years, different 
variants of autoencoder networks are employed for anomaly detection task
\cite{kingma2013autoencoding, Masci2011StackedCA, an2015variational, leveau2017adversarial, Pidhorskyi:2018:GPN:3327757.3327787}.

%% Modeli acikla cozmeye calistigin seyleri anlat
This thesis investigates the defect detection problem \cite{carrera2016defect} of 
SEM image dataset \cite{sem} (see Section \ref{sec:sem}). SEM images are images from the nanofibrous material manufacturing process called electro spinning method \cite{carrera2016defect}.
 Due to the environmental factors and failures related to the instruments, this method may produce 
samples which has defective/ anomalous regions (see Figure \ref{fig:data_samples}). 
While these anomalous regions have  
distinguishable features that can easily be spotted by human eye, it is difficult to 
catalogue these anomaly examples 
for detection with automated systems. Main contributing factors are that defective regions can have 
variety of appearence and shape and the formation of fibers also creates to difficulties for detection
since they can overlap and follow different orientations \cite{carrera2016defect}. Main approach to 
this problem is to build a model to learn the manifold of normal regions by using image patches 
which randomly extracted from the SEM images that has no defective regions. This model is then used 
to detect the defective regions of a new SEM image sample. Inference stage is performed by 
disecting the image into overlapping patches. This thesis provides a model that is trained with 
patches obtained from normal SEM images. Model performs a reconstruction based anomaly detection task 
by taking advantage of generative adversarial networks and encoder networks.

First part of thesis investigates the performance of anomaly detection methods that utilizes
generative adversarial networks (GAN) (see Section \ref{sec:gan}). Later on a modified model is proposed 
which consist of an energy based generative adversarial network and two encoder networks. 
Primary encoder network is trained adversarially with generator network to learn bidirectional 
mapping between the image space and latent dimension space. Secondary encoder network 
is trained to learn the latent representation of the reconstructed image samples obtained from 
the generator network. Comparative analysis between anomaly detection methods that uses 
generative adversarial networks and the proposed model regarding reconstruction quality 
of the generator network and overall performance is carried 
out via an ablation study and a series of incremental experiments (see Chapter \ref{chap:expres}).

Results of the experiments showed that training methodology obtained for the proposed model 
improved both the performance of generator network and encoder networks. It is also observed 
that anomaly detection performance of the model can be improved by utilizing an anomaly detection 
score based on the latent representations of the query images and their reconstructions rather 
than the images itself.

This document is organized as follows: Chapter \ref{chap:sota} presents the anomaly detection 
problem thoroughly, introduces the networks that are used for constructing anomaly 
detection methods such as generative adversarial networks and autoencoder networks and finally 
illustrates mainstream anomaly detection methods that utilizes generative adversarial 
networks. Chapter \ref{chap:latdel} introduces some of the latest developments concerning 
generative adversarial networks and anomaly detection task. Chapter \ref{chap:arim} 
analyzes the problems other GAN based anomaly detection models experience with SEM image dataset 
and presents our proposed model. Experimental results are described in Chapter \ref{chap:expres} 
and the conclusion and the future work are  stated in Chapter \ref{chap:conc}.

\endgroup