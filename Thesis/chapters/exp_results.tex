%!TEX root = ../2019_7_Ozgumus_Semsi_Yigit.tex

\begingroup

\section{Experiment Settings}

\subsection{Figure of Merit}
\section{GAN Based Model Analysis}
% Anogan

\begin{table}[]
	\centering
	\caption{Ablation study for AnoGAN to test the effect of various training improvements for stabilization.}
	\label{tab:anogan_ablation}
	\resizebox{\textwidth}{!}{%
		\begin{tabular}{|c|l|llll|}
			\hline
			\multicolumn{2}{|c|}{\multirow{2}{*}{\textbf{Model}}} & \multicolumn{4}{c|}{\textbf{Metrics}} \\ \cline{3-6} 
			\multicolumn{2}{|c|}{} & AUROC & \multicolumn{1}{c}{Precision} & \multicolumn{1}{c}{Recall} & \multicolumn{1}{c|}{F1 Score} \\ \hline
			\multirow{8}{*}{AnoGAN} & Normal &  &  &  &  \\ \cline{2-6} 
			& IN &  &  &  &  \\ \cline{2-6} 
			& SL &  &  &  &  \\ \cline{2-6} 
			& LF &  &  &  &  \\ \cline{2-6} 
			& IN + SL &  &  &  &  \\ \cline{2-6} 
			& IN + LF &  &  &  &  \\ \cline{2-6} 
			& SL + LF &  &  &  &  \\ \cline{2-6} 
			& LF + SL + IN &  &  &  &  \\ \hline
		\end{tabular}%
	}
\end{table}
% Bigan
 test
\begin{table}[]
	\centering
	\caption{Ablation study for BiGAN to test the effect of various training improvements for stabilization.}
	\label{tab:bigan_ablation}
	\resizebox{\textwidth}{!}{%
		\begin{tabular}{|c|l|llll|}
			\hline
			\multicolumn{2}{|c|}{\multirow{2}{*}{\textbf{Model}}} & \multicolumn{4}{c|}{\textbf{Metrics}} \\ \cline{3-6} 
			\multicolumn{2}{|c|}{} & AUROC & \multicolumn{1}{c}{Precision} & \multicolumn{1}{c}{Recall} & \multicolumn{1}{c|}{F1 Score} \\ \hline
			\multirow{8}{*}{BiGAN} & Normal &  &  &  &  \\ \cline{2-6} 
			& IN &  &  &  &  \\ \cline{2-6} 
			& SL &  &  &  &  \\ \cline{2-6} 
			& LF &  &  &  &  \\ \cline{2-6} 
			& IN + SL &  &  &  &  \\ \cline{2-6} 
			& IN + LF &  &  &  &  \\ \cline{2-6} 
			& SL + LF &  &  &  &  \\ \cline{2-6} 
			& LF + SL + IN &  &  &  &  \\ \hline
		\end{tabular}%
	}
\end{table}

% ALAD
\begin{table}[]
	\centering
	\caption{Ablation study for ALAD to test the effect of various training improvements for stabilization.}
	\label{tab:alad_ablation}
	\resizebox{\textwidth}{!}{%
		\begin{tabular}{|c|l|llll|}
			\hline
			\multicolumn{2}{|c|}{\multirow{2}{*}{\textbf{Model}}} & \multicolumn{4}{c|}{\textbf{Metrics}} \\ \cline{3-6} 
			\multicolumn{2}{|c|}{} & AUROC & \multicolumn{1}{c}{Precision} & \multicolumn{1}{c}{Recall} & \multicolumn{1}{c|}{F1 Score} \\ \hline
			\multirow{8}{*}{ALAD} & Normal &  &  &  &  \\ \cline{2-6} 
			& IN &  &  &  &  \\ \cline{2-6} 
			& SL &  &  &  &  \\ \cline{2-6} 
			& LF &  &  &  &  \\ \cline{2-6} 
			& IN + SL &  &  &  &  \\ \cline{2-6} 
			& IN + LF &  &  &  &  \\ \cline{2-6} 
			& SL + LF &  &  &  &  \\ \cline{2-6} 
			& LF + SL + IN &  &  &  &  \\ \hline
		\end{tabular}%
	}
\end{table}

\section{AR Based Model Analysis}

% GAnomaly
\begin{table}[]
	\centering
	\caption{Ablation study for GANomaly to test the effect of various training improvements for stabilization. }
	\label{tab:ganomaly_ablation}
	\resizebox{\textwidth}{!}{%
		\begin{tabular}{|c|l|llll|}
			\hline
			\multicolumn{2}{|c|}{\multirow{2}{*}{\textbf{Model}}} & \multicolumn{4}{c|}{\textbf{Metrics}} \\ \cline{3-6} 
			\multicolumn{2}{|c|}{} & AUROC & \multicolumn{1}{c}{Precision} & \multicolumn{1}{c}{Recall} & \multicolumn{1}{c|}{F1 Score} \\ \hline
			\multirow{8}{*}{GANomaly} & Normal &  &  &  &  \\ \cline{2-6} 
			& IN &  &  &  &  \\ \cline{2-6} 
			& SL &  &  &  &  \\ \cline{2-6} 
			& LF &  &  &  &  \\ \cline{2-6} 
			& IN + SL &  &  &  &  \\ \cline{2-6} 
			& IN + LF &  &  &  &  \\ \cline{2-6} 
			& SL + LF &  &  &  &  \\ \cline{2-6} 
			& LF + SL + IN &  &  &  &  \\ \hline
		\end{tabular}%
	}
\end{table}
% Skip Ganomaly

\begin{table}[]
	\centering
	\caption{Ablation study for Skip-GANomaly to test the effect of various training improvements for stabilization.}
	\label{tab:sganomaly_ablation}
	\resizebox{\textwidth}{!}{%
		\begin{tabular}{|c|l|llll|}
			\hline
			\multicolumn{2}{|c|}{\multirow{2}{*}{\textbf{Model}}} & \multicolumn{4}{c|}{\textbf{Metrics}} \\ \cline{3-6} 
			\multicolumn{2}{|c|}{} & AUROC & \multicolumn{1}{c}{Precision} & \multicolumn{1}{c}{Recall} & \multicolumn{1}{c|}{F1 Score} \\ \hline
			\multirow{8}{*}{Skip-GANomaly} & Normal &  &  &  &  \\ \cline{2-6} 
			& IN &  &  &  &  \\ \cline{2-6} 
			& SL &  &  &  &  \\ \cline{2-6} 
			& LF &  &  &  &  \\ \cline{2-6} 
			& IN + SL &  &  &  &  \\ \cline{2-6} 
			& IN + LF &  &  &  &  \\ \cline{2-6} 
			& SL + LF &  &  &  &  \\ \cline{2-6} 
			& LF + SL + IN &  &  &  &  \\ \hline
		\end{tabular}%
	}
\end{table}
\section{Improved Model Analysis}
\section{Discussion of Experiment Results}

This chapter will explain the experimental results for all the models and improvements. It will also
point out a discussion about what could be improved and the future direction.

There will be 3 classes of models to be considered. 
\begin{itemize}
    \item Models that maps $z$ to $x$ to find the image distrbituion and uses inverse sample for
    reconstruction $\rightarrow$ AnoGAN, BiGAN and ALAD
    \item Models that maps directly image distribution by integrating an encoder to the generator
    module hence creating in practise an autoencoder, and uses again, reconstruction $\rightarrow$
    Ganomaly and skip ganomaly
    \item Models that tries new methods to explore different solution strategies \begin{itemize}
        \item Training with full image $\rightarrow$ Segmentation papers
        \item Ganomaly + Noise addition (one Class paper concurrent work) to improve the performance
        of the image distribution learning.
        \item Energy Based GANS (loss function change), Can this method be applied to the encoder
        network as well, to better capture the noise distribution.
    \end{itemize}
\end{itemize}

\begin{itemize}
    \item All models will be tested with the standard improvements of the gan training. \begin{itemize}
        \item Label Flipping $\rightarrow$ improves gradints of the discriminator
        \item Soft Labels $\rightarrow$ Better than 0,1 reference the papers
        \item Adding noise to the input image and rectrated image to confuse discriminator
        $\rightarrow$ robustness
    \end{itemize}
    \item Best performance models then will be tested with a training with validation that is based
    on the reconstruction of the image. 
    \item All model results with ablation study
    \item Improved model results
    \item Visual Results like precision recall, AUC curve, Histogram of Anomalies
    \item Numerical Results like the result of the model performances in a table `†'
\end{itemize}

\endgroup