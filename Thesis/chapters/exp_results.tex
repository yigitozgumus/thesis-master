%!TEX root = ../2019_7_Ozgumus_Semsi_Yigit.tex

\begingroup

This chapter presents the performance of the gan based anomaly detection models and proposed model. 


\section{Experiment Settings}




\subsection{Performance Metrics}
This section will introduce the performance metrics used for the interpretation of the experiments performed.
These are:
\begin{itemize}
	\item Precision
	\item Recall
	\item F1 Score
	\item AUROC (Area Under Receiver Operating Characteristic curve)
\end{itemize}

 \textbf{Recall} is the ability of a model to find all the relevant cases within a dataset. In our case detection 
of all anomalies in a test would give us a recall of 1.0. \textbf{Precision} on the other hand is the ability 
ofa classification model to identify \textbf{only} the relevant data points. While recall expresses 
the ability to find all relevant instances in a dataset, precision expresses the proportion of the 
data points our model says was relevant actually were relevant. The calculation of the both metrics 
is given below
\begin{align}
\textbf{recall} & = \frac{\text{true positives}}{\text{true positives} + \text{false negatives}} \\[5pt]
\textbf{precision} & = \frac{\text{true positives}}{\text{true positives} + \text{false positives}}
\end{align}

Precision and recall comprises a trade off situation. If the model favors the precision, the recall 
decreases because eliminating the false positives inadvertently increases the false negative rate and 
vice versa. To give equal importance to both metrics, \textbf{F1 score} is used. The F1 score is the 
harmonic mean of precision and recall taking both metrics into account in the following equation:
\begin{equation}
\text{F}_{1} = 2 \times \frac{\text{precision} \times \text{recall}}{\text{precision} + \text{recall}}
\end{equation}

The last metric we use to interpret the model performance is area under receiver operating characteristic 
curve, \textbf{AUROC} for short. ROC curve visualizes the trade of relationship between the false positive 
and true positive rate. True positive rate is actually recall. False positive rate is the probability 
of false detection for the system. The calculations for these rates are provided below:
\begin{align}
\textbf{True Positive Rate} & = \frac{\text{true positives}}{\text{true positives} + \text{false negatives}} \\[5pt]
\textbf{False Positive Rate} & = \frac{\text{false positives}}{\text{true negatives} + \text{false positives}}
\end{align}

the AUROC value can be obtained by calculating the area under the ROC curve which has a range between 0 and 1 with a higher number indicating better classification performance.

\section{GAN Based Model Analysis}
% Anogan

\begin{table}[]
	\centering
	\caption{Ablation study for AnoGAN to test the effect of various training improvements for stabilization.}
	\label{tab:anogan_ablation}
	\resizebox{\textwidth}{!}{%
		\begin{tabular}{|c|l|llll|}
			\hline
			\multicolumn{2}{|c|}{\multirow{2}{*}{\textbf{Model}}} & \multicolumn{4}{c|}{\textbf{Metrics}} \\ \cline{3-6} 
			\multicolumn{2}{|c|}{} & AUROC & \multicolumn{1}{c}{Precision} & \multicolumn{1}{c}{Recall} & \multicolumn{1}{c|}{F1 Score} \\ \hline
			\multirow{8}{*}{AnoGAN} & Normal & \multicolumn{1}{c}{} & \multicolumn{1}{c}{} & \multicolumn{1}{c}{} & \multicolumn{1}{c|}{} \\ \cline{2-6} 
			& IN & \multicolumn{1}{c}{} & \multicolumn{1}{c}{} & \multicolumn{1}{c}{} & \multicolumn{1}{c|}{} \\ \cline{2-6} 
			& SL & \multicolumn{1}{c}{} & \multicolumn{1}{c}{} & \multicolumn{1}{c}{} & \multicolumn{1}{c|}{} \\ \cline{2-6} 
			& LF & \multicolumn{1}{c}{} & \multicolumn{1}{c}{} & \multicolumn{1}{c}{} & \multicolumn{1}{c|}{} \\ \cline{2-6} 
			& IN + SL & \multicolumn{1}{c}{} & \multicolumn{1}{c}{} & \multicolumn{1}{c}{} & \multicolumn{1}{c|}{} \\ \cline{2-6} 
			& IN + LF & \multicolumn{1}{c}{} & \multicolumn{1}{c}{} & \multicolumn{1}{c}{} & \multicolumn{1}{c|}{} \\ \cline{2-6} 
			& SL + LF & \multicolumn{1}{c}{} & \multicolumn{1}{c}{} & \multicolumn{1}{c}{} & \multicolumn{1}{c|}{} \\ \cline{2-6} 
			& LF + SL + IN & \multicolumn{1}{c}{} & \multicolumn{1}{c}{} & \multicolumn{1}{c}{} & \multicolumn{1}{c|}{} \\ \hline
		\end{tabular}%
	}
\end{table}
% Bigan
 test
\begin{table}[]
	\centering
	\caption{Ablation study for BiGAN to test the effect of various training improvements for stabilization.}
	\label{tab:bigan_ablation}
	\resizebox{\textwidth}{!}{%
		\begin{tabular}{|c|l|llll|}
			\hline
			\multicolumn{2}{|c|}{\multirow{2}{*}{\textbf{Model}}} & \multicolumn{4}{c|}{\textbf{Metrics}} \\ \cline{3-6} 
			\multicolumn{2}{|c|}{} & AUROC & \multicolumn{1}{c}{Precision} & \multicolumn{1}{c}{Recall} & \multicolumn{1}{c|}{F1 Score} \\ \hline
			\multirow{8}{*}{BiGAN} & Normal & \multicolumn{1}{c}{0.39614} & \multicolumn{1}{c}{} & \multicolumn{1}{c}{} & \multicolumn{1}{c|}{} \\ \cline{2-6} 
			& IN & \multicolumn{1}{c}{0.59329} & \multicolumn{1}{c}{} & \multicolumn{1}{c}{} & \multicolumn{1}{c|}{} \\ \cline{2-6} 
			& SL & \multicolumn{1}{c}{0.54633} & \multicolumn{1}{c}{} & \multicolumn{1}{c}{} & \multicolumn{1}{c|}{} \\ \cline{2-6} 
			& LF & \multicolumn{1}{c}{0.63394} & \multicolumn{1}{c}{} & \multicolumn{1}{c}{} & \multicolumn{1}{c|}{} \\ \cline{2-6} 
			& IN + SL & \multicolumn{1}{c}{0.53644} & \multicolumn{1}{c}{} & \multicolumn{1}{c}{} & \multicolumn{1}{c|}{} \\ \cline{2-6} 
			& IN + LF & \multicolumn{1}{c}{0.54526} & \multicolumn{1}{c}{} & \multicolumn{1}{c}{} & \multicolumn{1}{c|}{} \\ \cline{2-6} 
			& SL + LF & \multicolumn{1}{c}{0.57391} & \multicolumn{1}{c}{} & \multicolumn{1}{c}{} & \multicolumn{1}{c|}{} \\ \cline{2-6} 
			& LF + SL + IN & \multicolumn{1}{c}{0.36905} & \multicolumn{1}{c}{} & \multicolumn{1}{c}{} & \multicolumn{1}{c|}{} \\ \hline
		\end{tabular}%
	}
\end{table}

% ALAD
\begin{table}[]
	\centering
	\caption{Ablation study for ALAD to test the effect of various training improvements for stabilization.}
	\label{tab:alad_ablation}
	\resizebox{\textwidth}{!}{%
		\begin{tabular}{|c|l|llll|}
			\hline
			\multicolumn{2}{|c|}{\multirow{2}{*}{\textbf{Model}}} & \multicolumn{4}{c|}{\textbf{Metrics}} \\ \cline{3-6} 
			\multicolumn{2}{|c|}{} & AUROC & \multicolumn{1}{c}{Precision} & \multicolumn{1}{c}{Recall} & \multicolumn{1}{c|}{F1 Score} \\ \hline
			\multirow{8}{*}{ALAD} & Normal & \multicolumn{1}{c}{} & \multicolumn{1}{c}{} & \multicolumn{1}{c|}{} & \multicolumn{1}{c}{} \\ \cline{2-6} 
			& IN & \multicolumn{1}{c}{} & \multicolumn{1}{c}{} & \multicolumn{1}{c}{} & \multicolumn{1}{c|}{} \\ \cline{2-6} 
			& SL & \multicolumn{1}{c}{} & \multicolumn{1}{c}{} & \multicolumn{1}{c}{} & \multicolumn{1}{c|}{} \\ \cline{2-6} 
			& LF & \multicolumn{1}{c}{} & \multicolumn{1}{c}{} & \multicolumn{1}{c}{} & \multicolumn{1}{c|}{} \\ \cline{2-6} 
			& IN + SL & \multicolumn{1}{c}{} & \multicolumn{1}{c}{} & \multicolumn{1}{c}{} & \multicolumn{1}{c|}{} \\ \cline{2-6} 
			& IN + LF & \multicolumn{1}{c}{} & \multicolumn{1}{c}{} & \multicolumn{1}{c}{} & \multicolumn{1}{c|}{} \\ \cline{2-6} 
			& SL + LF & \multicolumn{1}{c}{} & \multicolumn{1}{c}{} & \multicolumn{1}{c}{} & \multicolumn{1}{c|}{} \\ \cline{2-6} 
			& LF + SL + IN & \multicolumn{1}{c}{} & \multicolumn{1}{c}{} & \multicolumn{1}{c}{} & \multicolumn{1}{c|}{} \\ \hline
		\end{tabular}%
	}
\end{table}

\section{AR Based Model Analysis}

% GAnomaly
\begin{table}[]
	\centering
	\caption{Ablation study for GANomaly to test the effect of various training improvements for stabilization. }
	\label{tab:ganomaly_ablation}
	\resizebox{\textwidth}{!}{%
		\begin{tabular}{|c|l|llll|}
			\hline
			\multicolumn{2}{|c|}{\multirow{2}{*}{\textbf{Model}}} & \multicolumn{4}{c|}{\textbf{Metrics}} \\ \cline{3-6} 
			\multicolumn{2}{|c|}{} & AUROC & \multicolumn{1}{c}{Precision} & \multicolumn{1}{c}{Recall} & \multicolumn{1}{c|}{F1 Score} \\ \hline
			\multirow{8}{*}{GANomaly} & Normal & \multicolumn{1}{c}{} & \multicolumn{1}{c}{} & \multicolumn{1}{c}{} & \multicolumn{1}{c}{} \\ \cline{2-6} 
			& IN & \multicolumn{1}{c}{} & \multicolumn{1}{c}{} & \multicolumn{1}{c}{} & \multicolumn{1}{c|}{} \\ \cline{2-6} 
			& SL & \multicolumn{1}{c}{} & \multicolumn{1}{c}{} & \multicolumn{1}{c}{} & \multicolumn{1}{c|}{} \\ \cline{2-6} 
			& LF & \multicolumn{1}{c}{} & \multicolumn{1}{c}{} & \multicolumn{1}{c}{} & \multicolumn{1}{c|}{} \\ \cline{2-6} 
			& IN + SL & \multicolumn{1}{c}{} & \multicolumn{1}{c}{} & \multicolumn{1}{c}{} & \multicolumn{1}{c|}{} \\ \cline{2-6} 
			& IN + LF & \multicolumn{1}{c}{} & \multicolumn{1}{c}{} & \multicolumn{1}{c}{} & \multicolumn{1}{c|}{} \\ \cline{2-6} 
			& SL + LF & \multicolumn{1}{c}{} & \multicolumn{1}{c}{} & \multicolumn{1}{c}{} & \multicolumn{1}{c|}{} \\ \cline{2-6} 
			& LF + SL + IN & \multicolumn{1}{c}{} & \multicolumn{1}{c}{} & \multicolumn{1}{c}{} & \multicolumn{1}{c|}{} \\ \hline
		\end{tabular}%
	}
\end{table}
% Skip Ganomaly

\begin{table}[]
	\centering
	\caption{Ablation study for Skip-GANomaly to test the effect of various training improvements for stabilization.}
	\label{tab:sganomaly_ablation}
	\resizebox{\textwidth}{!}{%
		\begin{tabular}{|c|l|llll|}
			\hline
			\multicolumn{2}{|c|}{\multirow{2}{*}{\textbf{Model}}} & \multicolumn{4}{c|}{\textbf{Metrics}} \\ \cline{3-6} 
			\multicolumn{2}{|c|}{} & AUROC & \multicolumn{1}{c}{Precision} & \multicolumn{1}{c}{Recall} & \multicolumn{1}{c|}{F1 Score} \\ \hline
			\multirow{8}{*}{Skip-GANomaly} & Normal & \multicolumn{1}{c}{} & \multicolumn{1}{c}{} & \multicolumn{1}{c}{} & \multicolumn{1}{c|}{} \\ \cline{2-6} 
			& IN & \multicolumn{1}{c}{} & \multicolumn{1}{c}{} & \multicolumn{1}{c}{} & \multicolumn{1}{c|}{} \\ \cline{2-6} 
			& SL & \multicolumn{1}{c}{} & \multicolumn{1}{c}{} & \multicolumn{1}{c}{} & \multicolumn{1}{c|}{} \\ \cline{2-6} 
			& LF & \multicolumn{1}{c}{} & \multicolumn{1}{c}{} & \multicolumn{1}{c}{} & \multicolumn{1}{c|}{} \\ \cline{2-6} 
			& IN + SL & \multicolumn{1}{c}{} & \multicolumn{1}{c}{} & \multicolumn{1}{c}{} & \multicolumn{1}{c|}{} \\ \cline{2-6} 
			& IN + LF & \multicolumn{1}{c}{} & \multicolumn{1}{c}{} & \multicolumn{1}{c}{} & \multicolumn{1}{c|}{} \\ \cline{2-6} 
			& SL + LF & \multicolumn{1}{c}{} & \multicolumn{1}{c}{} & \multicolumn{1}{c}{} & \multicolumn{1}{c|}{} \\ \cline{2-6} 
			& LF + SL + IN & \multicolumn{1}{c}{} & \multicolumn{1}{c}{} & \multicolumn{1}{c}{} & \multicolumn{1}{c|}{} \\ \hline
		\end{tabular}%
	}
\end{table}
\section{Improved Model Analysis}
\section{Discussion of Experiment Results}

This chapter will explain the experimental results for all the models and improvements. It will also
point out a discussion about what could be improved and the future direction.

There will be 3 classes of models to be considered. 
\begin{itemize}
    \item Models that maps $z$ to $x$ to find the image distrbituion and uses inverse sample for
    reconstruction $\rightarrow$ AnoGAN, BiGAN and ALAD
    \item Models that maps directly image distribution by integrating an encoder to the generator
    module hence creating in practise an autoencoder, and uses again, reconstruction $\rightarrow$
    Ganomaly and skip ganomaly
    \item Models that tries new methods to explore different solution strategies \begin{itemize}
        \item Training with full image $\rightarrow$ Segmentation papers
        \item Ganomaly + Noise addition (one Class paper concurrent work) to improve the performance
        of the image distribution learning.
        \item Energy Based GANS (loss function change), Can this method be applied to the encoder
        network as well, to better capture the noise distribution.
    \end{itemize}
\end{itemize}

\begin{itemize}
    \item All models will be tested with the standard improvements of the gan training. \begin{itemize}
        \item Label Flipping $\rightarrow$ improves gradints of the discriminator
        \item Soft Labels $\rightarrow$ Better than 0,1 reference the papers
        \item Adding noise to the input image and rectrated image to confuse discriminator
        $\rightarrow$ robustness
    \end{itemize}
    \item Best performance models then will be tested with a training with validation that is based
    on the reconstruction of the image. 
    \item All model results with ablation study
    \item Improved model results
    \item Visual Results like precision recall, AUC curve, Histogram of Anomalies
    \item Numerical Results like the result of the model performances in a table `†'
\end{itemize}

\endgroup